\date{\today}
\title{Joint Beamforming}
\author{Manu Thottathil Sreedharan}
\newcommand{\room}{ID2/348}
\newcommand{\tele}{23682}
\newcommand{\mail}{Manu.ThottathilSreedharan@rub.de}

\documentclass[]{dks-report}
\usepackage[hidelinks]{hyperref}
\usepackage{graphicx}
\usepackage{caption}
\usepackage{subcaption}
\usepackage{lipsum}
\begin{document}

\begin{center}
  \huge{\textbf{Joint Multi-User Beamforming}}
\end{center}

\section{Brief Description}
The project proposes to implement the idea described in \cite{Rahul2012_JMB}.
The paper describes signal processing techniques to achieve phase
synchronization required to implement a zero-forcing beamformer.
The goal of beamforming is to ensure that each client can decode its intended
signal without interference. In order to achieve phase synchornization in a
distributed fasion, elect
one AP as the master and other(hereafter called slave) APs have to maintain 
required phase offset with respect to the lead AP.

\section{Details}

\section{Hardware Details}
\begin{itemize}
  \item TX = X300 + 2 CBX(1200 - 6000MHz)
  \item RX = X300 + 2 WBX(50 - 2200 MHz)
  \item Operating Frequency = 1850MHz
\end{itemize}

\section{Requirements}

\bibliographystyle{ieeetr}
\bibliography{/home/sreedhm/Documents/references}

\end{document}
